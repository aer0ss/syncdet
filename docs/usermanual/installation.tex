\section{Installation}

\subsection{System Requirements}

Running SyncDET requires one or more UNIX, Linux, or cygwin systems.
The controller has only been tested for OSX and requires the
following:
\begin{enumerate}
\item Python 2.7+ installed
\item {\tt git}, {\tt tar}, {\tt rsync}, and {\tt grep}.
\item PyYaml, can be installed with {\tt pip install PyYaml}.
\end{enumerate}
The requirements for each actor follow:
\begin{enumerate}
\item Python 2.6+ installed.
\item An {\tt ssh} daemon server installed and running; future support can be
made for {\tt telnet} or {\tt rlogin} if desired. The controller should be able
to login without entering a password.
\item Commands {\tt ps}, {\tt grep}, {\tt tar}, {\tt rsync}, and {\tt sed}
installed. {\tt ps} must support the option {\tt -eo pid,cmd}, which lists all
processes with a PID and a command line per row.
\item PyYaml, can be installed with {\tt pip install PyYaml}.
\end{enumerate}

\subsection{Acquiring Source Code}
\begin{verbatim}
$ mkdir ~/repos; cd ~/repos
$ git clone ssh://g.arrowfs.org:44353/syncdet.git
\end{verbatim}

\subsection{Actor VMs}
SyncDET-ready virtual machines have been created for Linux, OSX Leopard, Windows
7 and Windows XP. They are shared as bzipped tarballs in an AeroFS library
somewhere (ask Mark or Weihan). They satisfy the system requirements listed
above, except for passwordless ssh; {\bf you must copy over your public key}. To
extract the linux VM, for example:
\begin{verbatim}
$ tar -xvf syncdet-ubuntu1110_32.tar.bz2 -C <parent directory of VM>
\end{verbatim}

\subsection{Systems Configuration}
On the controller system, first create a configuration file from the template:
\begin{verbatim}
$ sudo mkdir /etc/syncdet
$ cp syncdet/config.yaml.sample /etc/syncdet/config.yaml
\end{verbatim}

\textbf{NOTE} The above specified directory is simply a default path SyncDET checks for the
configuration file. You can specify your own configuration file path
by launching SyncDET with the \verb+--config+ flag as such:
\begin{verbatim}
$ syncdet.py --config=path/to/config.yaml
\end{verbatim}

Second, edit the configuration file to describe the actors. The file is
in the YAML format. The sample config file should show enough syntax to get you started.
For more information, see the YAML 1.1 spec at \verb+http://www.yaml.org/+.

\paragraph{Actors}
List the hostname or IP address of all systems to be used as actors:
\begin{verbatim}
# the dictionary definition of the actors
actors:
    -
        address: 192.168.1.16
    -
        address: 192.168.1.17
\end{verbatim}

\subsection{Running the Demo}

To run the entire scenario:
\begin{verbatim}
$ cd syncdet
$ ./syncdet.py examples/deploy -s examples/examples.scn
\end{verbatim}

Or, to run a single test case:
\begin{verbatim}
$ ./syncdet.py examples/deploy -c examples.hello_world
\end{verbatim}
