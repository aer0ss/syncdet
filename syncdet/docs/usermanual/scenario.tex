\section{SyncDET Scenarios}

From a controller (master) system, SyncDET deploys and executes distributed
scenarios on multiple actor (slave) systems. This section explains the structure
of a SyncDET scenario file, including the symbols and directives that constitute
a scenario.

\subsection{Concepts}
A scenario essentially represents the combination of a number of execution {\em
items}. An item can be a {\em test case} module, a group name, or even another
scenario name. Test cases, the smallest building blocks of SyncDET scenarios,
are small isolated Python functional tests. Using
{\em directives}, items can be executed in some particular combination (e.g.,
sequentially or in random order), and frequently-used scenarios can be combined
into larger scenarios.

Scenarios can be as simple or complex as you need; when writing your own SyncDET
scenario, you may only require the use of one to three test cases.

\subsection{Directives}
{\tt :serial}[,{\it count}]

{\tt :shuffle}[,{\it count}]
{\tt :parallel}[,{\it count}]

{\tt :opening}
{\tt :closing}
{\tt :scn[,nofail]} {\it name}
{\tt :group} {\it name}
{\tt :include} {\it path}

{\it name}{\tt ()}

\subsection{Use of Directives}
