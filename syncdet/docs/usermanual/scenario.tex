\section{Scenarios}

From a controller (master) system, SyncDET deploys and executes distributed
scenarios on multiple actor (slave) systems. This section explains the structure
of a SyncDET scenario file, including the symbols and directives that constitute
a scenario.

\subsection{Concepts}
A scenario essentially represents the combination of a number of execution {\em
items}. An item can be a {\em test case} module, a group name, or even another
scenario name. Test cases, the smallest building blocks of SyncDET scenarios,
are small isolated Python functional tests that either pass or fail. Each item
is executed concurrently on all actor systems, but {\em directives} are used to
define the execution order of items with respect to one system.  Items can be
executed in some particular combination (e.g., sequentially or in random order),
and frequently-used scenarios can be combined into larger scenarios.

Scenarios can be as simple or complex as you need; when writing your own SyncDET
scenario, you may only require the use of one to three test cases.

\subsection{Directives}
\directiveexec{serial}{count}

\directiveexec{repeat}{count}
Execute items sequentially. Each item will be executed sequentially count times
before proceeding to the next item. If count is zero, only the first item will
be executed, and will be executed infinite times until the program is
terminated. The default value for count is one. The two directives are
interchangeable.

\directiveexec{shuffle}{count}
Execute items sequentially, but in a random order. All items will be executed
exactly once unless {\it count} is non-zero, in which case {\it count} items
will be randomly selected and executed. If {\it count} is greater than the
number of items, some items will be executed more than once. The default value
for count is zero.

\directiveexec{parallel}{count}
Execute all items in the block in parallel (with respect to one system). Each
item will be copied count times and all copies will be executed in parallel with
copies of other items. {\it count} must not be zero. Its default value is one.

%\directivemarkup{opening}
%\directivemarkup{closing}

\directivemarkup[name]{scn[,nofail]}
Define a new scenario called {\it name}, which can be executed by calling {\it
name}{\tt ()} from within a scenario file. With {\tt nofail}, the entire
scenario is stopped if any item fails, otherwise a failed item is skipped and
the next item started.

%\directivemarkup[name]{group}

\directivemarkup[path]{include} Link to scenario file {\it path} for defined
scenarios.

\subsection{Use of Directives}

In this section, we list some typical uses of the directives. To loop over a
list of items for 10 times:
\begin{verbatim}
    :repeat,10
        :serial
            item1
            item2
            item3
\end{verbatim}
To execute a random item in a list:
\begin{verbatim}
    :shuffle,1
        item1
        item2
        item3
\end{verbatim}
To randomly pick ten items from a list, and execute all of them in parallel:
\begin{verbatim}
    :parallel,10
        :shuffle,1
            item1
            item2
            item3
\end{verbatim}
Note that in the above case, one item may have multiple instances executing in
parallel. We can define a reusable scenario for the first example that will
abort if any item fails, then execute it:
\begin{verbatim}
:scn,nofail firstEx
    :repeat,10
        :serial
            item1
            item2
            item3

:serial
    firstEx()
\end{verbatim}

