\section{Test Cases}
In this section, we provide further details on writing test cases, the building
blocks of scenarios. A test case is an ordinary Python script executed
concurrently by some number of actor systems. Systems are assigned an ID giving
them a total order. The following sections describe how to specify the
point-of-entry of the test case, how to write test functions, and the various
APIs to study.

\subsection{Sample Code}
For examples, see {\tt hello\_world.py} and {\tt sync.py} in the demo code.

\subsection{The {\tt spec} data structure}
Every test case python script must define a global dictionary, {\tt spec}, which
specifies the point-of-entry function to the test case across all actor systems,
and its permissable duration.  The following are the dictionary (key, value)
pairs, and their meanings.

\begin{center}
\begin{tabular}{ r | p{8cm} }
    key           & value \\
    \hline
    {\tt 'entries'} 
        & a list of length $n$, specifying the entry-point function names for
          the first $n$ systems \\
    %\hline
    {\tt 'default'}
        & the name of the entry-point function name for all other systems \\
    {\tt 'timeout'}
        & the maximum execution time of the test case, after which the test
          fails (integer, in seconds) \\
\end{tabular}
\end{center}

At least one of {\tt entries} or {\tt default} are required, and {\tt timeout}
is entirely optional, with the default specified in {\tt config.py}

\subsection{How test failures are detected}
A test case is considered a failure and will be reported in any of the 
following situations:

\begin{itemize}
\item An exception is raised and not caught within the entry-point function.
\item The test case process terminates before the entry-point function returns.
\item The entry-point function doesn't complete in time. (See CASE\_TIMEOUT
in config.py for details.)
\end{itemize}

Therefore, the simplest way to fail a test case is to raise an arbitrary
exception.

\subsection{Test Case API: {\tt import case}} 

This section briefly describes the general SyncDET API for writing test cases.
 
For functions to get information about the currently executing test case or
scenario see {\tt [SyncDETRoot]/syncdet/case/syncdet\_case\_lib.py}:
\begin{itemize}
\item get the local system ID, 
\item get the total number of systems executing the given test case, 
\item get the local SyncDET root (on the actor)
\item get the test case module name, and more.
\end{itemize}

Optionally, test-case-writers could have access to the systems.System class,
providing methods to scp files/folders to other actor systems. This is not
officially supported yet, but given any requests, it can be arranged.

\subsubsection{Synchronization API}

Three functions of interest are defined in \\*
{\tt [SyncDETRoot]/syncdet/case/syncdet\_case\_sync.py}:
\begin{itemize}
\item {\tt case.sync(...)}, a barrier across all systems
\item {\tt case.syncPrev(...)}, a barrier shared with this system and the previous
\item {\tt case.syncNext(...)}, a barrier shared with this system and the next
\end{itemize}
This API is also accessed through the {\tt case} module.

\subsection{AeroFS-specific Test API ({\tt aft} directory)}
Write your AeroFS-specific test scenarios and cases in the {\tt aft} directory.
An API of python functions will
\begin{itemize}
\item get AeroFS directories: Application, RunTime, File-sharing mount
\item launch AeroFS (assuming it is installed)
\item kill/terminate AeroFS
\item create a directory tree of files
\end{itemize}
in {\tt [SyncDETRoot]/aft/lib/lib.py}.

Also consider using the {\tt common.scn} scenario file for a Scenario interface
to
\begin{itemize}
\item {\tt start} AeroFS
\item {\tt stop} AeroFS
\end{itemize}

For both APIs, more functions will be written as required.

